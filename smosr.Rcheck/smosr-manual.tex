\nonstopmode{}
\documentclass[a4paper]{book}
\usepackage[times,inconsolata,hyper]{Rd}
\usepackage{makeidx}
\usepackage[utf8]{inputenc} % @SET ENCODING@
% \usepackage{graphicx} % @USE GRAPHICX@
\makeindex{}
\begin{document}
\chapter*{}
\begin{center}
{\textbf{\huge Package `smosr'}}
\par\bigskip{\large \today}
\end{center}
\inputencoding{utf8}
\ifthenelse{\boolean{Rd@use@hyper}}{\hypersetup{pdftitle = {smosr: A suite of tools to access, download and explore ESA-BEC Soil Moisture and Ocean Salinity (SMOS) data in R}}}{}
\begin{description}
\raggedright{}
\item[Title]\AsIs{A suite of tools to access, download and explore ESA-BEC Soil
Moisture and Ocean Salinity (SMOS) data in R}
\item[Version]\AsIs{0.1.0}
\item[Date]\AsIs{2023-07-13}
\item[Author]\AsIs{Tatiana A. Shestakova [aut, cre] (https://orcid.org/0000-0002-5605-0299)}
\item[Maintainer]\AsIs{Tatiana A. Shestakova }\email{tasha.work24@gmail.com}\AsIs{}
\item[Description]\AsIs{Provides functions that automate accessing, downloading and importing ESA-BEC
Soil Moisture and Ocean Salinity (SMOS) data into R. Particularly, it includes functions to 
search for, acquire, extract, and plot BEC-SMOS L4 soil moisture data downscaled to ~1 km 
spatial resolution (EASE-grid v.2).}
\item[URL]\AsIs{}\url{https://github.com/tshestakova/smosr}\AsIs{}
\item[License]\AsIs{GPL-3}
\item[Encoding]\AsIs{UTF-8}
\item[Depends]\AsIs{R (>= 4.0.0)}
\item[RoxygenNote]\AsIs{7.2.3}
\item[Imports]\AsIs{fields, graphics, grDevices, lubridate, methods, ncdf4,
raster, RCurl, terra, tidyr, utils}
\item[Suggests]\AsIs{knitr, rmarkdown, testthat (>= 3.0.0)}
\item[VignetteBuilder]\AsIs{knitr}
\item[Config/testthat/edition]\AsIs{3}
\item[NeedsCompilation]\AsIs{no}
\item[BugReports]\AsIs{}\url{https://github.com/tshestakova/smosr/issues}\AsIs{}
\end{description}
\Rdcontents{\R{} topics documented:}
\inputencoding{utf8}
\HeaderA{download\_smos}{Download BEC-SMOS soil moisture data}{download.Rul.smos}
%
\begin{Description}
This function automates downloading of original BEC-SMOS soil moisture data
to a local computer via a secure FTP (SFTP) server.
\end{Description}
%
\begin{Usage}
\begin{verbatim}
download_smos(data, dir = NULL)
\end{verbatim}
\end{Usage}
%
\begin{Arguments}
\begin{ldescription}
\item[\code{data}] a character vector as produced by \code{find\_smos()} containing
external links to the data files on the BEC server.

\item[\code{dir}] a character string specifying a path to a local directory in which
to save the data. Default value is \code{NULL} which means the dataset is
stored in the current working directory.
\end{ldescription}
\end{Arguments}
%
\begin{Details}
This function downloads the original BEC-SMOS soil moisture data in NetCDF
format ("as is") from the BEC server. The data files are stored on the local
computer in the current working directory (default option) if no otherwise
specified by a user.

Note that this function requires a username and a password to access the BEC
server. Pass your credentials in using the
\code{\LinkA{set\_credentials()}{set.Rul.credentials}} function. Otherwise, if you
do not have your BEC login details yet, please register on
\url{https://bec.icm.csic.es/bec-ftp-service-registration}.
\end{Details}
%
\begin{Examples}
\begin{ExampleCode}
## Not run: 
# to download all files found with find_smos()
# into the current working directory
start_date <- as.Date("2022-01-01")
end_date <- as.Date("2022-12-31")
date_range <- seq(start_date, end_date, by = 30)
smos_data <- find_smos(freq = 3, orbit = "descending", dates = date_range)
download_smos(smos_data)
# to download first five items from the complete list of files
# and place them in a newly created directory
dir.create("~/SMOS_data")
download_smos(data = smos_data[1:5], dir = "~/SMOS_data")

## End(Not run)

\end{ExampleCode}
\end{Examples}
\inputencoding{utf8}
\HeaderA{extract\_smos}{Extract BEC-SMOS soil moisture estimates for specific geographical locations}{extract.Rul.smos}
%
\begin{Description}
This function facilitates reading the original BEC-SMOS soil moisture data
files and extracting relevant information for specific geographical locations
by using Lat/Lon coordinates in decimal degrees.
\end{Description}
%
\begin{Usage}
\begin{verbatim}
extract_smos(data, lat, lon, save = FALSE, filename = "SM_output.csv")
\end{verbatim}
\end{Usage}
%
\begin{Arguments}
\begin{ldescription}
\item[\code{data}] a character vector as produced by \code{list\_smos()} containing
links to the data files on the local computer.

\item[\code{lat}] a numeric vector containing latitudes of points to extract the
data from (in 'latlon' projection).

\item[\code{lon}] a numeric vector containing longitudes of points to extract the
data from (in 'latlon' projection).

\item[\code{save}] a logical vector indicating whether the output should  be saved
as a CSV file. Default is \code{FALSE}.

\item[\code{filename}] a character string naming a file for saving the output. If
\code{save} = \code{TRUE} and no \code{filename} is specified by the user,
the data is saved in a file with a generic name 'SM\_output.csv'.
\end{ldescription}
\end{Arguments}
%
\begin{Details}
This function reads the original BEC-SMOS soil moisture data files in NetCDF
format, converts data from EASE-2 grid cells to geographic coordinates, and
extracts relevant information for Lat/Lon locations specified by the user.

The data retrieved from each data file includes:

- the coordinates of spatial points (Lon and Lat) from which the data were
extracted;

- frequency and SMOS orbit of each file over which the function iterated;

- date and time when the data was obtained;

- soil moisture estimate (SM);

- quality assurance (QA) corresponding to each SM estimate. Good quality data
is marked with QA = 0. To know the meanings of QA > 0, please refer to the
technical note on the BEC-SMOS soil moisture products available at
\url{https://digital.csic.es/handle/10261/303808}.

The output of this function could be saved as a CSV file.
\end{Details}
%
\begin{Value}
a data.matrix with the relevant information as described in Details.
\end{Value}
%
\begin{Examples}
\begin{ExampleCode}
## Not run: 
# to iterate over BEC-SMOS data files stored in the current working directory
# and extract soil moisture estimates for the specified geographical locations
smos_data <- list_smos()
lat <- c(40.42, 41.90, 48.86, 52.50, 59.91)
lon <- c(-3.70, 12.50, 2.35, 13.40, 10.75)
sm_estimates <- extract_smos(data = smos_data, lat = lat, lon = lon)

## End(Not run)

\end{ExampleCode}
\end{Examples}
\inputencoding{utf8}
\HeaderA{find\_smos}{Find BEC-SMOS soil moisture data in Barcelona Expert Center (BEC) server}{find.Rul.smos}
%
\begin{Description}
This function searches for BEC-SMOS soil moisture data available on Barcelona
Expert Center (BEC) server for the frequency, orbit, and dates specified by
the user.
\end{Description}
%
\begin{Usage}
\begin{verbatim}
find_smos(freq, orbit, dates)
\end{verbatim}
\end{Usage}
%
\begin{Arguments}
\begin{ldescription}
\item[\code{freq}] an integer specifying temporal frequency of the data. Possible
values are: 1 - for daily data, or 3 - for 3-day moving averages. No
default value is provided.

\item[\code{orbit}] a character (or character string) specifying SMOS orbit
corresponding to the data. Possible values are: ‘a’, ‘asc’, and ‘ascending’ -
for an ascending pass, or ‘d’, ‘des’, or ‘descending’ - for a descending
pass. No default value is provided.

\item[\code{dates}] an object of class \code{Date} or a character string formatted
as ‘yyyy-mm-dd’ (e.g. ‘2010-06-01’) which specifies the date(s) to search
through. To look for a specific date, it can be a Date object or a character
vector of length 1. To iterate over various dates or a time interval, a
multiple-element object of class Date or a vector should be passed (e.g. as
produced by \code{seq.Date}).
\end{ldescription}
\end{Arguments}
%
\begin{Details}
BEC-SMOS soil moisture (SM) data is a regional root zone SM product that
covers Europe and Mediterranean countries. Particularly, \code{smosr} package
works with the reprocessed level-4 (L4) SM estimates downscaled to \textasciitilde{}1 km
spatial resolution (EASE-grid v.2). The data is computed for two time periods
(argument \code{frequency}): daily and 3-day moving averages produced by a
temporal aggregation of the daily products. Note that SMOS ascending and
descending passes (argument \code{orbit}) are processed separately. The
BEC-SMOS SM product is available starting from June 1st, 2010 throughout the
end of 2022. The currently supported version is 6.0. For more details about
the BEC-SMOS soil moisture products, see the technical note available at
\url{https://digital.csic.es/handle/10261/303808}.

Note that this function requires a username and a password to access the BEC
server. Pass your credentials in using the
\code{\LinkA{set\_credentials()}{set.Rul.credentials}} function. Otherwise, if you
do not have your BEC login details yet, please register on
\url{https://bec.icm.csic.es/bec-ftp-service-registration}.
\end{Details}
%
\begin{Value}
a character vector containing full links to the data files on the
server.
\end{Value}
%
\begin{Examples}
\begin{ExampleCode}
## Not run: 
# to look for SMOS data on a specific date
smos_data <- find_smos(freq = 1, orbit = "a", dates = "2022-12-31")
# to search over a date range
start_date <- as.Date("2022-01-01")
end_date <- as.Date("2022-12-31")
date_range <- seq(start_date, end_date, by = 10)
smos_data <- find_smos(freq = 3, orbit = "descending", dates = date_range)

## End(Not run)

\end{ExampleCode}
\end{Examples}
\inputencoding{utf8}
\HeaderA{list\_smos}{List the BEC-SMOS data files stored on a local computer}{list.Rul.smos}
%
\begin{Description}
This function returns a list of the BEC-SMOS data files previously stored on
a local computer.
\end{Description}
%
\begin{Usage}
\begin{verbatim}
list_smos(
  freq = NULL,
  orbit = NULL,
  dates = NULL,
  dir = NULL,
  recursive = FALSE
)
\end{verbatim}
\end{Usage}
%
\begin{Arguments}
\begin{ldescription}
\item[\code{freq}] an integer specifying temporal frequency of the data. Possible
values are: 1 - for daily data, or 3 - for 3-day moving averages, and NULL -
for cases when data frequency is irrelevant. Default value is \code{NULL}.

\item[\code{orbit}] a character (or character string) specifying SMOS orbit
corresponding to the data. Possible values are: ‘a’, ‘asc’, and ‘ascending’ -
for an ascending pass, or ‘d’, ‘des’, or ‘descending’ - for a descending
pass, and NULL - for cases when orbit is irrelevant. Default value is
\code{NULL}.

\item[\code{dates}] an object of class \code{Date} or a character string formatted
as ‘yyyy-mm-dd’ (e.g. ‘2010-06-01’) which specifies the date(s) to search
through. To look for a specific date, it can be a Date object or a character
vector of length 1. To iterate over various dates or a time interval, a
multiple-element object of class Date or a vector should be passed (e.g. as
produced by \code{seq.Date}).

\item[\code{dir}] a character string specifying a path to a local directory in which
to search for the data. Default value is \code{NULL} which means the dataset
is looked up in the current working directory.

\item[\code{recursive}] a logical vector indicating whether the listing should
recurse into directories. Default is \code{FALSE}.
\end{ldescription}
\end{Arguments}
%
\begin{Details}
This function iterates over all files in the current working directory
(default option) or another local folder as indicated by \code{dir} argument
and returns a list of the BEC-SMOS data files with the \code{frequency},
\code{orbit}, and \code{dates} specified by the user. If no arguments are
provided, all BEC-SMOS soil moisture data files found in the selected folder
will be listed. A recursive option is also available.
\end{Details}
%
\begin{Value}
a character vector containing full links to the data files on the
local computer.
\end{Value}
%
\begin{Examples}
\begin{ExampleCode}
## Not run: 
# to list all BEC-SMOS data files stored in the current working directory
# as well as in the corresponding subfolders
smos_data <- list_smos(recursive = TRUE)
# to list BEC-SMOS data files with the specified frequency and SMOS orbit
# stored in the specified folder
smos_data <- list_smos(freq = 3, orbit = "asc", dir = "~/SMOS_data")

## End(Not run)

\end{ExampleCode}
\end{Examples}
\inputencoding{utf8}
\HeaderA{missing\_smos}{Print the dates for which BEC-SMOS soil moisture data have not been found}{missing.Rul.smos}
%
\begin{Description}
This function prints out the dates for which BEC-SMOS soil moisture data with
specified \code{frequency} and \code{orbit} arguments have not been found on
the BEC server. This information is automatically generated while running
\code{\LinkA{find\_smos()}{find.Rul.smos}}, but displayed only if requested by the
user.
\end{Description}
%
\begin{Usage}
\begin{verbatim}
missing_smos()
\end{verbatim}
\end{Usage}
%
\begin{Value}
a character string containing dates for which the data files were not
found on the server.
\end{Value}
%
\begin{Examples}
\begin{ExampleCode}
## Not run: 
missing_smos()

## End(Not run)

\end{ExampleCode}
\end{Examples}
\inputencoding{utf8}
\HeaderA{plot\_raster\_smos}{Draw a raster image of BEC-SMOS soil moisture data}{plot.Rul.raster.Rul.smos}
%
\begin{Description}
This function draws a raster image of BEC-SMOS soil moisture data
corresponding to a single data file and specific geographical extent.
\end{Description}
%
\begin{Usage}
\begin{verbatim}
plot_raster_smos(data, lat = NULL, lon = NULL, QA = NULL)
\end{verbatim}
\end{Usage}
%
\begin{Arguments}
\begin{ldescription}
\item[\code{data}] a character string containing a link to a single BEC-SMOS data
file stored on the local computer.

\item[\code{lat}] a numeric vector of length 2 containing latitudinal bounds of
the plotting region (in 'latlon' projection). Default value is \code{NULL}
which means all data between min and max latitudes are drawn.

\item[\code{lon}] a numeric vector of length 2 containing longitudinal bounds of
the plotting region (in 'latlon' projection). Default value is \code{NULL}
which means all data between min and max longitudes are drawn.

\item[\code{QA}] a numeric vector specifying the desired quality assurance of the
data to be drawn. Possible values range from 0 (good quality data) to 15.
To know the meanings of QA > 0, please refer to the technical note on the
BEC-SMOS soil moisture products available at
\url{https://digital.csic.es/handle/10261/303808}.
\end{ldescription}
\end{Arguments}
%
\begin{Details}
This function reads an original BEC-SMOS soil moisture data file in NetCDF
format, converts data from EASE-2 grid cells to geographic coordinates, and
draws a raster image of soil moisture estimates in 'latlon' projection. The
image can be drawn for a specific geographical extent if requested by the
user. Otherwise, the entire dataset across Europe (between 28 and 72 degrees
north and -11 and 40 degrees east) will be plotted.

Note that due to high resolution of the data (\textasciitilde{}1 km), the execution of this
function may take a long time to be completed depending on the amount of
data to be drawn.
\end{Details}
%
\begin{Examples}
\begin{ExampleCode}
## Not run: 
# to draw a raster image of soil moisture data within specified bounds
smos_data <- list_smos()
lat <- c(35.00, 45.00)
lon <- c(-10.50, 4.50)
plot_raster_smos(data = smos_data[1], lat = lat, lon = lon)

## End(Not run)

\end{ExampleCode}
\end{Examples}
\inputencoding{utf8}
\HeaderA{plot\_temporal\_smos}{Plot temporal series of BEC-SMOS soil moisture data}{plot.Rul.temporal.Rul.smos}
%
\begin{Description}
This function plots temporal series of BEC-SMOS soil moisture data
extracted for specific geographical locations using \code{extract\_smos()}.
\end{Description}
%
\begin{Usage}
\begin{verbatim}
plot_temporal_smos(data, freq = NULL, orbit = NULL, dates = NULL, QA = NULL)
\end{verbatim}
\end{Usage}
%
\begin{Arguments}
\begin{ldescription}
\item[\code{data}] a data.matrix as produced by \code{list\_smos()} containing soil
moisture data to plot.

\item[\code{freq}] an integer specifying temporal frequency of the data. Possible
values are: 1 - for daily data, 3 - for 3-day moving averages, and NULL -
for cases when data frequency is irrelevant. Default value is \code{NULL}.

\item[\code{orbit}] a character (or character string) specifying SMOS orbit
corresponding to the data. Possible values are: ‘a’, ‘asc’, and ‘ascending’ -
for an ascending pass, or ‘d’, ‘des’, or ‘descending’ - for a descending
pass, and NULL - for cases when orbit is irrelevant. Default value is
\code{NULL}.

\item[\code{dates}] a object of class \code{Date} or a character string formatted
as ‘yyyy-mm-dd’ (e.g. ‘2010-06-01’) which specifies the dates to plot the
data for.

\item[\code{QA}] a numeric vector specifying the desired quality assurance of the
data to be plotted. Possible values range from 0 (good quality data) to 15.
To know the meanings of QA > 0, please refer to the technical note on the
BEC-SMOS soil moisture products available at
\url{https://digital.csic.es/handle/10261/303808}.
\end{ldescription}
\end{Arguments}
%
\begin{Details}
Note that the data characterized by the same frequency and SMOS orbit can
be drawn at a time. If the dataset to plot contains a mixture of temporal
resolution and/or SMOS passes, arguments \code{frequency} and \code{orbit}
must be specified.
\end{Details}
%
\begin{Examples}
\begin{ExampleCode}
## Not run: 
# to plot extracted temporal series of soil moisture data
# with the specified frequency, SMOS orbit and QA
smos_data <- list_smos()
lat <- c(40.42, 41.90, 48.86, 52.50, 59.91)
lon <- c(-3.70, 12.50, 2.35, 13.40, 10.75)
sm_estimates <- extract_smos(data = smos_data, lat = lat, lon = lon)
plot_temporal_smos(data = sm_estimates, freq = 3, orbit = "d", QA = 0)

## End(Not run)

\end{ExampleCode}
\end{Examples}
\inputencoding{utf8}
\HeaderA{set\_credentials}{Set credentials to access Barcelona Expert Center (BEC) server}{set.Rul.credentials}
%
\begin{Description}
To use some functionalities of \code{smosr} package (e.g. access the server
or download data to a local computer), the user should first register at
Barcelona Expert Center (BEC) webpage. This function allows the authenticated
users to set their BEC credentials (username and password) for the current R
session which are used internally in \code{\LinkA{find\_smos()}{find.Rul.smos}}
and \code{\LinkA{download\_smos()}{download.Rul.smos}}.
\end{Description}
%
\begin{Usage}
\begin{verbatim}
set_credentials(username, password)
\end{verbatim}
\end{Usage}
%
\begin{Arguments}
\begin{ldescription}
\item[\code{username}] a character string containing BEC server username.

\item[\code{password}] a character string containing BEC server password.
\end{ldescription}
\end{Arguments}
%
\begin{Details}
If you do not have your BEC login details yet, please register on
\url{https://bec.icm.csic.es/bec-ftp-service-registration}.
\end{Details}
%
\begin{Value}
The function returns a character string with the inputs pasted
together in the format required by \code{\LinkA{find\_smos()}{find.Rul.smos}}
and \code{\LinkA{download\_smos()}{download.Rul.smos}}.
\end{Value}
%
\begin{Examples}
\begin{ExampleCode}
## Not run: 
set_credentials("username", "password")

## End(Not run)

\end{ExampleCode}
\end{Examples}
\printindex{}
\end{document}
